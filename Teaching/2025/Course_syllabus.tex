\documentclass[a4paper,11pt]{article}


\usepackage{amsmath,amsgen,amstext,amssymb}
\usepackage{amsfonts,amsopn,amsthm}
\usepackage{dsfont}
\usepackage{mathrsfs}
\usepackage{latexsym}
\usepackage{times}
\usepackage{graphics}
\usepackage{graphicx}
\usepackage{epsfig}
\usepackage{psfrag}
\usepackage{bm}
\usepackage{algorithm}
\usepackage{verbatim}
\usepackage{mathtools}
\usepackage{subcaption}
\usepackage{authblk}
\usepackage{hyperref}
\usepackage[left=0.9in,top=1.2in,right=0.9in]{geometry}
\usepackage{qcircuit}
\usepackage{color}

\DeclareMathAlphabet{\mathpzc}{OT1}{pzc}{m}{it}
%\DeclareMathOperator*{\argmax}{\arg\max}
%\DeclareMathOperator*{\argmin}{\arg\min}
\DeclarePairedDelimiter\bra{\langle}{\rvert}
\DeclarePairedDelimiter\ket{\lvert}{\rangle}
\DeclarePairedDelimiterX\braket[2]{\langle}{\rangle}{#1 \delimsize\vert #2}
\DeclarePairedDelimiterX\dotp[2]{\langle}{\rangle}{#1, #2}
\DeclareMathOperator{\diag}{diag}


\newtheorem{theorem}{Theorem}[section]
\newtheorem{proposition}{Proposition}[section]
\newtheorem{lemma}{Lemma}[section]
\newtheorem{corollary}{Corollary}[section]
\newtheorem{assumption}{Assumption}[section]
\newtheorem{definition}{Definition}[section]
\newtheorem{remark}{Remark}[section]
\newtheorem{claim}{Claim}[section]
\newtheorem{conjecture}{Conjecture}[section]

\newcommand{\bsigma}{\bm{\sigma}}
% \newcommand{\bS}{\boldsymbol{s}}
\newcommand{\bq}{\mathbf{q}}
\newcommand{\brho}{\bm{\rho}}
\newcommand{\bmu}{\bm{\mu}}
\newcommand{\bnu}{\bm{\nu}}
\newcommand{\bomega}{\bm{\omega}}

\newcommand{\K}{\mathcal K}
\newcommand{\R}{\mathbb R}
%\newcommand{\I}{\mathcal I}
\newcommand{\cX}{\mathcal X}
\newcommand{\J}{\mathcal J}

\newcommand{\E}{\mathbb{E}}
\newcommand{\ex}{\mathbb{E}}
\newcommand{\pr}{\mathbb{P}}
\newcommand{\ud}{\mathrm{d}}
\newcommand{\var}{\operatorname{Var}}
\newcommand{\norm}[1]{\| #1 \|}
\newcommand{\indic}{\mathbb{I}}

\newcommand{\beqn}{\begin{equation}}
\newcommand{\eeqn}{\end{equation}}


\newcommand{\sd}{\mbox{\sf sd}}
\newcommand{\cv}{\mbox{\sf Cov}}
\newcommand{\pdf}{\mbox{\sf pdf}}

\newcommand{\mode}[1]{{\mathit{Mo}}( #1 )}
\newcommand{\bpi}{{\bm{\pi}}}
\newcommand\Reals{{\mathbb{R}}}
\newcommand\Ints{{\mathbb{Z}}}
\newcommand{\Nats}{{\mathbb{N}}}
\newcommand{\eqd}{{\stackrel{d}{=}}}
%\newcommand{\eqdef}{{\triangleq}}
\newcommand{\eqdef}{{:=}}
\providecommand{\abs}[1]{{\lvert#1\rvert}}
\providecommand{\card}[1]{{\lvert#1\rvert}}
\providecommand{\norm}[1]{{\lVert#1\rVert}}
\newcommand{\ind}{{\mathbf{1}}}
\newcommand{\zero}{{\mathbf{0}}}

\newcommand{\bA}{{\mathbf{A}}}
\newcommand{\bB}{{\mathbf{B}}}
\newcommand{\bC}{{\mathbf{C}}}
\newcommand{\bD}{{\mathbf{D}}}
\newcommand{\bF}{{\mathbf{F}}}
\newcommand{\bG}{{\mathbf{G}}}
\newcommand{\bH}{{\mathbf{H}}}
\newcommand{\bI}{{\mathbf{I}}}
\newcommand{\bK}{{\mathbf{K}}}
\newcommand{\bL}{{\mathbf{L}}}
\newcommand{\bM}{{\mathbf{M}}}
\newcommand{\bP}{{\mathbf{P}}}
\newcommand{\bQ}{{\mathbf{Q}}}
\newcommand{\bR}{{\mathbf{R}}}
\newcommand{\bT}{{\mathbf{T}}}
\newcommand{\bU}{{\mathbf{U}}}
\newcommand{\bV}{{\mathbf{V}}}
\newcommand{\bW}{{\mathbf{W}}}
\newcommand{\bX}{{\mathbf{X}}}
\newcommand{\bY}{{\mathbf{Y}}}
\newcommand{\bZ}{{\mathbf{Z}}}
\newcommand{\balpha}{{\bm{\alpha}}}
\newcommand{\ba}{{\mathbf{a}}}
\newcommand{\bb}{{\mathbf{b}}}
\newcommand{\bc}{{\mathbf{c}}}
\newcommand{\bfe}{{\mathbf{e}}}
\newcommand{\Bf}{{\mathbf{f}}}
\newcommand{\Bm}{{\mathbf{m}}}
\newcommand{\bn}{{\mathbf{n}}}
\newcommand{\br}{{\mathbf{r}}}
\newcommand{\bu}{{\mathbf{u}}}
\newcommand{\bv}{{\mathbf{v}}}
\newcommand{\bw}{{\mathbf{w}}}
\newcommand{\bx}{{\mathbf{x}}}
\newcommand{\bs}{{\mathbf{s}}}
\newcommand{\bS}{{\bm{S}}}
\newcommand{\by}{{\mathbf{y}}}
\newcommand{\bz}{{\mathbf{z}}}
\newcommand{\bt}{{\mathbf{t}}}
\newcommand{\bbC}{{\mathbb{C}}}
\newcommand{\bbD}{{\mathbb{D}}}
\newcommand{\bbH}{{\mathbb{H}}}
\newcommand{\bbK}{{\mathbb{K}}}
\newcommand{\bbM}{{\mathbb{M}}}
\newcommand{\bbS}{{\mathbb{S}}}

\newcommand{\tr}{\text{Tr}}
\newcommand{\defeq}{\mbox{$\triangleq$}}

\newcommand{\al}{\alpha}                %%
\newcommand{\bet}{\beta}                %%
\newcommand{\g}{\lambda}                %%
\newcommand{\ga}{\gamma}                %% abbreviated
\newcommand{\dt}{\delta}                %% greek letters
\newcommand{\Dt}{\Delta}                %% greek letters
\newcommand{\la}{\lambda}               %%
\newcommand{\lam}{\lambda}               %%
\newcommand{\Lam}{\Lambda}               %%
\newcommand{\sig}{\sigma}               %%
\newcommand{\s}{\sigma}               %%
\newcommand{\om}{\omega}                %%
%\newcommand{\th}{\theta}                %%
%\newcommand{\ee}{\epsilon}                %%

\newcommand{\evt}{{\cal E}}             %% event list

\newcommand{\cA}{{\mathcal{A}}}
\newcommand{\cB}{{\mathcal{B}}}
\newcommand{\cC}{{\mathcal{C}}}
\newcommand{\cF}{{\mathcal{F}}}
\newcommand{\cG}{{\mathcal{G}}}
\newcommand{\cH}{{\mathcal{H}}}
%\newcommand{\cI}{{\mathcal{I}}}
\newcommand{\cJ}{{\mathcal{J}}}
\newcommand{\cK}{{\mathcal{K}}}
\newcommand{\ck}{{\mathpzc{k}}}
\newcommand{\cL}{{\mathcal{L}}}
\newcommand{\cl}{{\mathpzc{l}}}
\newcommand{\cM}{{\mathcal{M}}}
\newcommand{\cN}{{\mathcal{N}}}
\newcommand{\cP}{{\mathcal{P}}}
\newcommand{\cR}{{\mathcal{R}}}
\newcommand{\cS}{{\mathcal{S}}}
\newcommand{\cV}{{\mathcal{V}}}

%\newcommand{\calstoi}{{\cal S}_{\to i}}
\newcommand{\calstoi}{{\cal S}_{> i}}
\newcommand{\calsjto}{{\cal S}_{j>}}
%\newcommand{\calsito}{{\cal S}_{i\to}}
\newcommand{\calsito}{{\cal S}_{i >}}

\newcommand{\ra}{\rightarrow}           %%
\newcommand{\Ra}{\Rightarrow}           %% arrows
\newcommand{\imp}{\Rightarrow}           %% arrows
\newcommand{\callRa}{\calleftrightarrow}      %%
\newcommand{\subs}{\subseteq}           %% subset or equal to
\newcommand{\stle}{\le_{\rm st}}        %% stochastically less than
\newcommand{\abk}{{\mathbf{(a,b,k)}}}

\newcommand{\hti}{{\tilde h}}
\newcommand{\tL}{{\tilde L}}
\newcommand{\wh}{\widehat}
\newcommand{\rinv}{R^{\mbox{\small -\tiny 1}}}   %% a nicer R^{-1}
\newcommand{\pre}{\preceq}
\newcommand{\Bbar}{\overline{B}}
\newcommand{\ub}{\underline{b}}
\newcommand{\starti}{\parindent0pt\it}  %% start an italic line
\newcommand{\startb}{\parindent0pt\bf}  %% start a boldface line
%\newcommand{\mc}{\multicolumn}  %% multicolumn used in tables
\newcommand{\lrge}{\ge_{\rm lr}}        %%
\newcommand{\lrle}{\le_{\rm lr}}        %%
%\newcommand\1{\mathds{1}}
%\newcommand{\indi}[1]{\1_{\{#1\}}}
\newcommand{\indi}[1]{\1 {\{#1\}}}
\newcommand{\IndSet}{\mathcal{I}}
\newcommand{\Int}{\mathbb{Z}}
\newcommand{\bzero}{\mathbf{0}}
\newcommand{\ones}{\mathbf{e}}
\newcommand{\vectop}{\mathrm{vec}}
\newcommand{\dft}{\mathrm{DFT}}
\newcommand{\idft}{\mathrm{IDFT}}

%\def\ba#1\ea{\begin{align*}#1\end{align*}}
%\def\ban#1\ean{\begin{align}#1\end{align}}
%\newcommand{\norm}[1]{\left\lVert#1\right\rVert}
%\newcommand{\todo}[1]{ \textcolor{blue}{\textsc{!!ToDo:} #1 !! }}
\newcommand{\todo}[1]{}

\newcommand{\bE}{\ensuremath{\mathds{E}} }
\newcommand{\bEq}[1]{ \mathds{E} \left[\left. #1\right|\bf{q}(t)=\bf{q}\right]  }
\newcommand{\bone}{\ensuremath{\mathds{1}} }

\newcommand{\vq}{\ensuremath{{\bf q}} }
\newcommand{\bvq}{\ensuremath{\overline{{\bf q}}} } %Steady State
\newcommand{\va}{\ensuremath{{\bf a}} }
\newcommand{\bva}{\ensuremath{\overline{{\bf a}}} } %Steady State
%\newcommand{\vs}{\ensuremath{{\bf s}} }
\newcommand{\vu}{\ensuremath{{\bf u}} }
\newcommand{\valpha}{\ensuremath{{\mbox{\boldmath{$\alpha$}}}} }
\newcommand{\vlam}{\ensuremath{{\mbox{\boldmath{$\lambda$}}}} }
\newcommand{\vsig}{\ensuremath{{\mbox{\boldmath{$\sigma$}}}} }
\newcommand{\vnu}{\ensuremath{{\mbox{\boldmath{$\nu$}}}} }
\newcommand{\vzeta}{\ensuremath{{\mbox{\boldmath{$\zeta$}}}} }
\newcommand{\vchi}{\ensuremath{{\mbox{\boldmath{$\chi$}}}} }
\newcommand{\vk}{\ensuremath{{\mbox{\boldmath{$k$}}}} }
\newcommand{\vqt}{\ensuremath{{\bf q}(t)} }
\newcommand{\vqtn}{\ensuremath{{\bf q}(t+1)} }
\newcommand{\vx}{\ensuremath{{\bf x}} }
\newcommand{\vy}{\ensuremath{{\bf y}} }

\newcommand{\ve}{\ensuremath{{\bf e}} } %Can change any of these is need be.
\newcommand{\vet}{\ensuremath{\widetilde{{\bf e}}} }
\newcommand{\vei}{\ensuremath{{\bf e}^{(i)}} }
\newcommand{\vetj}{\ensuremath{\widetilde{{\bf e}}^{(j)}} }
\newcommand{\ei}{\ensuremath{e^{(i)}} }
\newcommand{\etj}{\ensuremath{\widetilde{ e}^{(j)}} }
\newcommand{\ki}{\ensuremath{\kappa^{(i)}} }
\newcommand{\ktj}{\ensuremath{\widetilde{ \kappa}^{(j)}} }

\newcommand{\vone}{\ensuremath{{\bf 1}} } % all one vector

\newcommand{\qij}{\ensuremath{q_{ij}} }
\newcommand{\bqij}{\ensuremath{\overline{q}_{ij}} } %Steady State
\newcommand{\qijt}{\ensuremath{q_{ij}(t)} }
%\newcommand{\bq}{\ensuremath{\overline{q}} }
%\newcommand{\qijt1}{\ensuremath{q_{ij}(t+1)} }
\newcommand{\aij}{\ensuremath{a_{ij}} }
\newcommand{\baij}{\ensuremath{\overline{a}_{ij}} } %Steady State
\newcommand{\sij}{\ensuremath{s_{ij}} }
\newcommand{\uij}{\ensuremath{u_{ij}} }
\newcommand{\lij}{\ensuremath{_{ij}} }
%\newcommand{\lijl}{\ensuremath{_{ij'}} } %using l for ' in the macro
%\newcommand{\lilj}{\ensuremath{_{i'j}} }
%\newcommand{\lil}{\ensuremath{_{i'}} }
%\newcommand{\ljl}{\ensuremath{_{j'}} }
\newcommand{\para}{\ensuremath{_{\parallel}} }
\newcommand{\per}{\ensuremath{_{\perp}} }
\newcommand{\paraij}{\ensuremath{_{\parallel ij}} }
\newcommand{\perij}{\ensuremath{_{\perp ij}} }
\newcommand{\se}{\ensuremath{^{(\epsilon)}} }
\newcommand{\numin}{\ensuremath{\nu_{\min}} }
\newcommand{\numinp}{\ensuremath{\nu_{\min}'} }
\newcommand{\rt}{\ensuremath{\widetilde{r}} }

\newcommand{\cKo}{\ensuremath{\mathcal{K}^{\circ}} }

\newcommand{\N}{{\cal N}}
\newcommand{\call}{{\cal L}}
\newcommand{\calm}{{\cal M}}
\newcommand{\calb}{{\cal B}}
\newcommand{\calc}{{\cal C}}
\newcommand{\calp}{{\cal P}}
\newcommand{\calg}{{\cal G}}
\newcommand{\calk}{{\cal K}}
\newcommand{\cals}{{\cal S}}
\newcommand{\calI}{{\cal I}}
\newcommand{\calJ}{{\cal J}}

\newcommand{\cala}{{\cal A}}
\newcommand{\calf}{{\cal F}}
\newcommand{\caln}{{\cal N}}

\definecolor{blue}{rgb}{0,0,1}
\definecolor{red}{rgb}{1,0,0}
\definecolor{green}{rgb}{.5,.8,.5}
\definecolor{noidea}{rgb}{.5,.4,0}
\newcommand\new[1]{\textcolor{blue}{#1}}

    
\title{CSE 392 – Topics in Computer Science: Matrix and Tensor Algorithms for Data}

\author{Instructor: Shashanka Ubaru}

\date{}

\begin{document}
\thispagestyle{empty}

\maketitle


\section*{Course Description}
Advances in modern technologies have resulted in huge volumes of data being generated in several scientific, industrial, and social domains. With ever increasing size of
data comes the necessity to develop fast and scalable machine learning and data algorithms to process and analyze them. In this course, we study the mathematical foundations of large-scale data processing, with focus on  designing algorithms and learning to (theoretically) analyze them. We explore randomized numerical linear algebra (sketching and sampling) and tensor methods for processing and analyzing large-scale multidimensional data, graphs, and data-streams. We will also have presentations on the linear algebra concepts of quantum computing. 
\section*{ Prerequisites}
The minimum requirements for the course are basics concepts of probability, algorithms, and linear algebra. Knowledge and experience with machine learning algorithms will be helpful. For the course, we will rely most heavily on probability, linear and tensor algebra, but we will also learn few concepts related to approximation theory, high dimensional geometry, and  quantum computing. The course will involve rigorous theoretical analyses and some programming (practical implementation and applications). \\


\noindent{\bf Programming language: } The  programming languages for the course will be  \emph{Matlab} and \emph{Python}.

\section*{Grading}
Grading is based on problem sets,  project/presentation, and class participation. There will be no exams.  The breakdown is as follows:
\begin{itemize}
\item ($50\%$)  4 to 5 assignments each contributing an equal amount to the grade. Assignments will include problem sets and programming exercises. 
\item  ($40\%$) Class project and presentation.
\item ($10\%$) Participation in the class.
\end{itemize}

\section*{Resources}
There is no official textbook for the class. Course material will mainly consist of lecture notes/slides, along with online resources such as papers, notes from other courses, and publicly available surveys. 

\section*{Course Schedule}

The following schedule is tentative and subject to change.
\begin{itemize}
\item Week 1 - {\bf Introduction and basics}
\begin{itemize}
\item Lecture 1: Vector spaces, matrices, norms.
\item Lecture 2:  Probability review, concentration of measure.
\end{itemize}

\item Week 2  - {\bf Regression and low rank approximation}
\begin{itemize}
\item Lecture 3: Least squares regression, kernel methods.
\item Lecture 4:  Matrix factorizations I - Singular Value Decomposition (SVD),  QR factorization.
\end{itemize}

\item Week 3 -  {\bf Matrix factorization and Randomized projection}
\begin{itemize}
\item Lecture 5: Matrix factorizations II - eigenvalue decomposition, Principal Component Analysis.
\item Lecture 6: Approximate matrix product, sampling. 
\end{itemize}

\item Week 4  - {\bf Randomized Sketching}
\begin{itemize}
\item Lecture 7: Johnson–Lindenstrauss(JL) lemma, subspace embedding. 
\item Lecture 8: Sketching, types of sketching matrices. 
\end{itemize}

\item Week 5  - {\bf Randomized linear algebra I}
\begin{itemize}
\item Lecture 9: Sketch and solve - least squares regression. 
\item Lecture 10: Sampling for least squares, preconditioned LS. 
\end{itemize}


\item Week 6  - {\bf Randomized linear algebra II}
\begin{itemize}
\item Lecture 11: Randomized SVD.
\item Lecture 12: Stochastic trace estimation.
\end{itemize}

\item Week 7 - {\bf Iterative methods}
\begin{itemize}
\item Lecture 13 :  Subspace iteration (power) method.
\item Lecture 14 :  Krylov subspace method.
\end{itemize}

\item Week 8  - {\bf Tensor methods - CP foundations}
\begin{itemize}
\item Lecture 15:  Introduction to tensors, tensor-matrix product.
\item Lecture 16:  Canonical Polyadic (CP) decomposition.
\end{itemize}

\item Week 9   - {\bf Randomized CP decomposition}
\begin{itemize}
\item Lecture 17: Kronecker Fast JL, randomized CP-ALS.
\item Lecture 18: CP-ALS with leverage scores, Generalized CP.
\end{itemize}

\item Week 10  - {\bf Tucker Decomposition}
\begin{itemize}
\item Lecture 19: Tucker decomposition, HOSVD.
\item Lecture 20: Randomized Tucker.
\end{itemize}

\item Week 11  - {\bf Matrix mimetic tensor algebra I}
\begin{itemize}
\item Lecture 21: Tube-fiber product, t-product.
\item Lecture 22:  Tensor-tensor-SVD (t-SVD).
\end{itemize}

\item Week 12  - {\bf Matrix mimetic tensor algebra II}
\begin{itemize}
\item Lecture 23: Randomized t-SVD, t-product applications.
\item Lecture 24: Tensor networks.
\end{itemize}

\item Week 13  - {\bf Quantum computing} (Optional)
\begin{itemize}
\item Lecture 25: Introduction to quantum computing, vector states, Pauli matrices, tensor product,
\item Lecture 26:   Quantum  circuits and quantum measurements.
\end{itemize}

\item Week 14  - {\bf Presentations}
\begin{itemize}
\item Lecture 27: Project presentations I
\item Lecture 28: Project presentations II
\end{itemize}


\end{itemize}

The quantum computing topics (Week 13) above will be discussed as presentations, provided we are able to cover other topics as per schedule. There will not be any problem sets related to these topics.
\end{document}
